\documentclass[12pt]{article}
\usepackage{amsmath}

\begin{document}

\section{Simplifying Fractions}


Siimplifying Fractions means re writing the fraction using the 
smallest top and bottom nummbers that we can, without changing the value of 
the fraction 

\subsection{excercise}
$$ \text{simplyfi } 6 - 2 \div 7 + 1 $$

write as fraction

$$ 6- \frac{2}{7} +1 $$

claculate parts together

$$ 7 - \frac{2}{7} $$

bring on the same denominator

$$ \frac{7}{1} - \frac{2}{7} = \frac{49}{7} - \frac{2}{7} = \frac{47}{7} $$

\subsection{excercise}

$$ \text{Simplify } 2 \cdot 7 - 7 + 4 $$

calculate

$$ 14 -7 + 4 = 7+4 = 11 $$

\subsection{excercise}

Question: what is the value of
$$ 1 \div 5 + 5 \cdot 5 $$

write as fraction
$$ \frac{1}{5} + 5 \cdot 5 $$

evaluate multiplication
$$ \frac{1}{5} + 25 $$

bring on one determinator
$$ \frac{1}{5} + \frac{125}{5} = \frac{126}{5}$$

\subsection{$ \text{Simplify } 3 \div 5 + 9 \cdot 2 $}

write as fraction
$$ \frac{3}{5} + 9 \cdot 2 $$

evaluate $ 9 \cdot 2 $
$$ \frac{3}{5} + 18 $$

bring on the same denominator, and evaluate result
$$ \frac{3}{5} + \frac{90}{5}  = \frac{93}{5}$$

\end{document}