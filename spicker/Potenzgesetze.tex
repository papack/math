\documentclass[11pt]{article}

\begin{document}


\section{Potenzgesetze}


Potenz mit Exponent 0 is immer 1
$$ a^{0} = 1 $$

Potenz mit dem Exponent 1 enfällt
$$ a^{1} = a $$ 

Multiplikation von Potenzen mit gleicher Basis: Potenzen mit gleicher Basis werden multipliziert, indem ihre Exponenten addiert werden.
$$ a^{m} \cdot a^{n} = a^{m+n} $$

Potenzierung von Potenzen: Potenzen werden potenziert, indem alle Exponenten miteinander multipliziert werden.
$$ (a^{m})^{n} = a^{m \cdot n} $$

Multiplikation von Potenzen mit gleichem Exponent: Potenzen mit gleichem Exponent werden multipliziert, indem die Basen multipliziert werden.
$$ a^{n} \cdot b^{n} = (ab)^{n} $$

Potenz mit negativem Exponenten
$$ a^{-n} = \frac{1}{a^{n}}$$

Division von Potenzen mit gleicher Basis
$$ \frac{a^{n}}{a^{m}} = a^{n-m} $$ 

Potenz deren Exponent das Inverse einer natürlichen Zahl ist
$$ a^{\frac{1}{n}} = \sqrt[n]{a}$$

Potenz deren Exponent ein Bruch ist. (Achtung: wenn n gerade ist, muss a größer als 0 sein!)
$$ a^{\frac{m}{n}} = \sqrt[n]{a^{m}} $$

\end{document}
